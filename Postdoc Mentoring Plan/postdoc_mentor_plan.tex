\documentclass[10pt]{SelfArx}

\PassOptionsToPackage{usenames,dvipsnames}{xcolor}
%!TEX root = ./2017_NSF_proposal.tex
\AtBeginDocument{%
  \addtolength\abovedisplayskip{-0.5\baselineskip}%
  \addtolength\belowdisplayskip{-0.5\baselineskip}%
%  \addtolength\abovedisplayshortskip{-0.5\baselineskip}%
%  \addtolength\belowdisplayshortskip{-0.5\baselineskip}%
}



\usepackage{fec}
\usepackage[sort&compress]{natbib}
 \usepackage{comment}
\usepackage{graphicx}
\usepackage{multirow}
\usepackage{multicol}

% \RequirePackage{algorithmic,algorithm}




\newcommand{\tr}{\textrm{tr}}

\newcommand{\inTR}[1]{}
\newcommand{\setN}{N}

\usepackage{algorithm}
% \usepackage{algorithmic}
\usepackage{algpseudocode}

\usepackage[colorinlistoftodos,bordercolor=orange,backgroundcolor=orange!20,linecolor=orange,textsize=scriptsize]{todonotes}


%\usepackage[disable]{todonotes}



%\JournalInfo{\today} % Journal information
%\Archive{XXXX} % Additional notes (e.g. copyright, DOI, review/research article)


%\Authors{}
%\Keywords{} % Keywords - if you don't want any simply remove all the text between the curly brackets
\newcommand{\keywordname}{Keywords} % Defines the keywords heading name
%\Abstract{}


%\usepackage{subfigure}
%\usepackage{subfig}
\usepackage{enumitem}

\usepackage{epsfig}
\usepackage{color}
\usepackage{boxedminipage}
\usepackage{amsfonts}
\usepackage{amsmath}
\usepackage{url}

\usepackage{epstopdf}
\usepackage{enumitem}
\usepackage{natbib}
\usepackage{pdfpages}
\newcommand{\note}[1]{{\small \color{red} \it #1}}



\setlength{\bibsep}{0pt plus 0.3ex}
\setlength{\columnsep}{0.55cm} % Distance between the two columns of text
\setlength{\fboxrule}{0.75pt} % Width of the border around the abstract


%\usepackage{fontspec}
%\setmainfont{Arial}

 \newtheorem{theorem}{Theorem}
 \newtheorem{lemma}{Lemma}
% \newtheorem{propres}{Proposed Research}
%\newtheorem{propres}{Research Topic}
\newtheorem{propres}{Framework}

%---------------------------------------------------------------------------------------
%   COLORS
%----------------------------------------------------------------------------------------
\definecolor{color1}{HTML}{81071c} %   0,0,90  Color of the article title and sections
\definecolor{color2}{RGB}{0.972549,0.8352941, 0.5137255} % Color of the boxes behind the abstract and headings
\definecolor{light-gray}{gray}{0.1}
\color{light-gray}


%----------------------------------------------------------------------------------------
%   ABSTRACT
%----------------------------------------------------------------------------------------



\usepackage{epsfig}
\usepackage{epstopdf}

%----------------------------------------------------------------------------------------
%   HYPERLINKS
%----------------------------------------------------------------------------------------
% \usepackage{hyperref} % Required for hyperlinks

\newcommand\tagthis{\addtocounter{equation}{1}\tag{\theequation}}


\usepackage{amsmath}
\DeclareMathOperator{\Rand}{Rand}
\DeclareMathOperator{\support}{support}



% revision and comment
\usepackage[normalem]{ulem}
\usepackage[dvipsnames]{xcolor}
\newcommand{\del}[1]{{\ignorespaces \color{red} \sout{#1}}}
\newcommand{\add}[1]{{\color{blue} {#1}}}
\newcommand{\rewrite}[1]{{\color{SeaGreen} {#1}}}
\newcommand{\tored}[1]{{\color{red} {#1}}}



                       % such that
\newcommand{\ve}[2]{\left\langle #1 ,  #2 \right\rangle}    % inner product
\newcommand{\eqdef}{:=}
                    % set of real numbers
\newcommand{\Prob}{\mathbb{P}}                   % probability
                       % expectation

%\newcommand{\U}{U}
\newcommand{\argmin}{argmin}

\newcommand{\R}{ {\bf R}}
\newcommand{\Var}{\mathbf{Var}}
\newcommand{\hatZ}{\hat Z}
\newcommand{\hatS}{\hat S}
\newcommand{\calG}{G}
\newcommand{\calT}{\mathcal{T}}
\newcommand{\calF}{\mathcal{F}}
\newcommand{\Lip}{\mathcal{L}}
\newcommand{\calH}{\mathcal{H}_\beta}
\newcommand{\calHMINI}{\mathcal{C}}

\newcommand{\calS}{\mathcal{S}}
\newcommand{\calJ}{\mathcal{J}}
\newcommand{\x}{ {x}}
\newcommand{\y}{ {\bf y}}
\newcommand{\z}{ {\bf z}}
\newcommand{\vv}{ {\bf v}}
\newcommand{\wv}{ {\bf w}}
\newcommand{\av}{ {\bf \alpha}}
\newcommand{\bv}{ {\bf \alpha}}
\newcommand{\V}{ {\bf v}}
\newcommand{\T}{ {\bf T}}
\newcommand{\X}{ {\bf X}}

\newcommand{\0}{ {\bf 0}}
\newcommand{\alf}{ {\boldsymbol \alpha}}
\newcommand{\vchi}{ {\boldsymbol \chi}}

\newcommand{\vu}{{\bf u}}
\newcommand{\vdelta}{{\boldsymbol \delta} }
\newcommand{\w}{  {\bf w}}
\newcommand{\vt}{  {\bf t}}
\newcommand{\va}{  {\bf a}}

\newcommand{\f}{f}
\newcommand{\YY}{\varphi}
\newcommand{\K}{K}
\newcommand{\sK}{\mathcal{K}}
\newcommand{\J}{J}
\newcommand{\sJ}{\mathcal{J}}

\newcommand{\calO}{\mathcal{O}}
\newcommand{\vsubset}[2]{#1_{[#2]}}
\newcommand{\Srv}{\hat{S}}
\newcommand{\oo}{|J \cap \Srv_j|}

\newcommand{\vc}[2]{#1^{(#2)}}                   % coordinate of a vector
\newcommand{\cor}[2]{{{#1}_{#2}}}                   % coordinate of a vector
\newcommand{\corit}[3]{{#1}_{#2}^{(#3)}}                   % coordinate of a vector
\newcommand{\ncs}[2]{\|#1\|^2_{(#2)}}
\newcommand{\nbp}[2]{\|#1\|_{(#2)}}              % norm block primal
\newcommand{\nbd}[2]{\|#1\|_{(#2)}^*}            % norm block dual
\newcommand{\Rw}[2]{\mathcal R_{#1}(#2)}
\newcommand{\Rws}[2]{\mathcal R^2_{#1}(#2)}
        % Cardinality

\newcommand{\pnote}[1]{{  \color{red} [[ #1 -- Peter ]] }}
\newcommand{\natinote}[1]{{  \color{blue} [[ #1 -- Nati ]] }}
\newcommand{\takinote}[1]{{  \color{yellow} [[ #1 -- Martin ]] }}
\newcommand{\avnote}[1]{{  \color{green} [[ #1 -- Avleen ]] }}
\newcommand{\removed}[1]{}
\newcommand{\norm}[1]{\left\lVert{#1}\right\rVert}
\newcommand{\hingeloss}{\ell}
\newcommand{\hinge}[1]{\hingeloss ( #1 )}
\newcommand{\trans}{{\top}}
\newcommand{\calA}{\mathcal{A}}
\newcommand{\bP}{\mathcal{P}}
\newcommand{\bD}{\mathcal{D}}
\newcommand{\bH}{\mathcal{H}}
\newcommand{\sizeJ}{\omega}
\newcommand{\Gg}{\mathcal{G}^{\sigma'}}


\newcommand{\newstuff}[1]{{\color{red}#1}}
\newcommand{\cocoa}{\textsc{CoCoA}}
\newcommand{\cocoap}{\textsc{CoCoA$\!^{\bf \textbf{\footnotesize+}}$}}
\newcommand{\localalgname}{\textsc{LocalSolver}\xspace}
\newcommand{\localSDCA}{\textsc{LocalSDCA}\xspace}
\newcommand{\setn}{[n]}
\newcommand{\Exp}{\mathbb{E}}                      % expectation


\usepackage{caption}
\usepackage{subcaption}
%----------------------------------------------------------------------------------------
%   HYPERLINKS
%----------------------------------------------------------------------------------------
\usepackage{hyperref} % Required for hyperlinks
\hypersetup{colorlinks,breaklinks=true,urlcolor=color2,citecolor=color1,linkcolor=color1,bookmarksopen=false,pdftitle={Title},pdfauthor={Author}}
\usepackage{algorithm}
\usepackage{wrapfig}
\usepackage{hyperref}
\hypersetup{colorlinks,urlcolor=blue}

 
%\PaperTitle{Postdoctoral Researcher Mentoring Plan}
 
%**********
% Document
%**********
\begin{document}

\pagestyle{plain}

\todo[inline{\large THIS HAS TO BE MODIFIED, it is from Lehigh}}


\begin{center}
{\color{color1} \bf \large
Postdoctoral Researcher Mentoring Plan}
\end{center}

During this project, a postdoctoral researcher (``postdoc'') will be employed at Lehigh University.  This Postdoctoral Researcher Mentoring Plan describes the recruitment plan and mentoring guidelines used for all postdoctoral positions with the Optimization and Machine Learning (\href{http://optml.lehigh.edu/}{OptML @ Lehigh}) research group.

\paragraph{Recruitment and Orientation.} Postdocs are recruited through an open recruiting process. Postdoc candidates will be interviewed by one or more investigators. At interview time, mutual expectations will be established for a) the amount of independence the postdoc will have, b) interactions with other team members, c) productivity, including the importance of scientific publications, d) work habits, and e) documentation of research methodologies and experimental details so that work can be continued by other researchers.

\paragraph{Professional Responsibility.} After joining the team, postdocs are expected to immediately take and pass any required courses on research conduct and research ethics. Postdocs will also be encouraged to join professional societies if they did not already join as graduate students.

\paragraph{Research Mentoring.} Postdocs will be included in the intellectual leadership of the research. They will be encouraged to develop their own unique lines of research and to produce first-author publications of this work. They will also be included in discussions with industrial partners, thus providing them with exposure to the practical and intellectual shaping of the research agenda. Postdocs will also be involved in the development of research proposals in this area. This will include identification of key research questions, definition of objectives, description of approach and rationale, and formulation of a work plan, timeline, and budget. Finally, postdocs will be instructed and included in technology transfer activities including applicable confidentiality requirements and preparation of invention disclosure applications, as applicable.

\paragraph{Teaching Mentoring.} Postdocs will be encouraged to participate in teaching and mentoring. All members of the research team are active in teaching, and teach courses in topic areas that would be relevant to any postdoc employed on this project. Postdocs will be encouraged to deliver lectures in these courses and they may, if appropriate, develop their own courses. They will also work closely with the graduate students on the project and will thus gain valuable mentoring experience of their own.  At least one of the investigators will provide the postdoc yearly detailed evaluations on course related material, such as course lecture slides, course assignments, and lecture presentations.

\paragraph{Career Counseling.} Each postdoc will be mentored with the skills, knowledge, and experience needed to excel in their chosen career path. This guidance will be provided by all investigators on this proposal.  Indeed, the sharing of postdocs allows each postdoc to create a larger career mentoring network than would otherwise be possible.

\paragraph{Assessment.} The success of this Mentoring Plan will be assessed in multiple ways.  During the course of the project, the postdoc will maintain an individual development plan, which will be regularly revisited and reviewed by both the postdoc and investigators.  Of key importance will be whether the goals discussed with the postdoc are being achieved, and to what level. The investigators will also personally monitor the postdoc's progress toward their career goals.

\end{document}