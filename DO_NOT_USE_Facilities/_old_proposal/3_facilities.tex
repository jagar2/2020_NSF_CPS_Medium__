
\pdfoutput=1
 
\documentclass[12pt]{article}

 \usepackage[margin=1.2in]{geometry}



\usepackage{epsfig}
\usepackage{color}
\usepackage{boxedminipage}
\usepackage{amsfonts}
\usepackage{amsmath}
\usepackage{url}
%\usepackage[hmargin=1in,vmargin=1in]{geometry}
\usepackage{enumitem}
\usepackage{natbib}
\usepackage{pdfpages}
 
%\setlength{\bibsep}{0pt plus 0.3ex}
%\setlength{\columnsep}{0.55cm} % Distance between the two columns of text
%\setlength{\fboxrule}{0.75pt} % Width of the border around the abstract

%----------------------------------------------------------------------------------------
%	COLORS
 

%----------------------------------------------------------------------------------------
%	HYPERLINKS
%----------------------------------------------------------------------------------------
\usepackage{hyperref} % Required for hyperlinks
 
 

\begin{document}



 \begin{center}
 \bf
  Facilities, Equipment, and Other Resources
 \end{center}
  
 
 
The laboratory for Computational Optimization Research at Lehigh (COR@L)
\url{http://coral.ise.lehigh.edu/}
will be available for prototyping, benchmarking, and software development.
It has a cluster consisting of 15 CPU nodes, each node has 16 cores and  32GB of RAM
and one GPU node with two nVidia K80 GPUs.
The cluster has many software/tools for debugging and profiling.

 


To perform larger-scale 
testing, Lehigh University's central High Performance Computing facilities will also be utilized.
Lehigh university has  
{\bf Corona Cluster}
with 
64 nodes,
each node has 2 AMD Opteron 8-core processors (hence 16 cores/node) and 32GB RAM, resulting in
1,024 cores. A large scale storage for preprocessing the testing dataset is also available.

We will also utilize the high-performance cluster at Siemens when working with proprietary data.


%We also anticipate the opportunity to work with Dheevatsa Mudigere and Mikhail Smelyanskiy at Intel and with Jakub Marecek at IBM to be able to test our methods on real systems and applications.  







\end{document}